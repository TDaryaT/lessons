Полное описание функции состоит из (то, как строят сложные функции):
\begin{enumerate}
    \item Нахождение области определения функции $D(y)$
    \item Четноть, нечетность, периодичность \\
    Функция называется \textbf{чётной}, если справедливо равенство
    $\displaystyle f(-x) = f(x)$, (график её симметричен относительно центра координат). \\
    Функция называется \textbf{нечётной}, если справедливо равенство
    $\displaystyle f(-x)=-f(x)$, (график её симметричен относительно оси ординат) \\
    \textbf{Ни чётная, ни нечётная функция} такие тоже бывают :) \\
    \textbf{Периодическая функция} ― функция, повторяющая свои значения через некоторый регулярный интервал аргумента, то есть не меняющая своего значения при добавлении к аргументу некоторого фиксированного ненулевого числа (периода функции) на всей области определения. 
    \item Непрерывность
    \item Асимптоты \\
    \textbf{Асимптота} - прямая, обладающая тем свойством, что расстояние от точки кривой до этой прямой стремится к нулю при удалении точки вдоль ветви в бесконечность. \\
    По простому - то, к чему функция стремится, но не может достичь
    \item Нули функции и интервалы знакопостоянства
    \item Интервалы монотонности и экстремумы \\
    \textbf{Монотонность} - возрастание/убывание функции; \\
    \textbf{Экстремум} - максимальное или минимальное значение функции на заданном множестве (то есть на границах не может быть экстремума!). Точка, в которой достигается экстремум, называется точкой экстремума
    \item Выпуклость. Вогнутость. Точка перегиба
    \item Дополнительные точки
    \item Область значений функции $E(y)$
    \item График функции
\end{enumerate}

Пример полного описания функции будет в конце (Приложение 1) 

\subsection{Задание}
Провести полное исследование функции и построить график функции:

\[
    y = \frac{x^2}{4(x+2)}
\]